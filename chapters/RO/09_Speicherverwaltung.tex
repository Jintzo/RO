\section{Virtuelle Speicherverwaltung}

\textbf{Notwendigkeit}
\begin{items}
  \item Immer größere Programme
  \item Immer mehr Programme "`gleichzeitig"'
  \item \( \leadsto \) verfügbarer Arbeitsspeicher schnell zu klein
  \item \underline{Lösung}: Nur gerade benötigte Teile der aktiven Programme im Arbeitsspeicher, Rest bei Bedarf aus Hintergrundspeicher nachladen (\emph{swapping}, \emph{paging})
  \item \underline{Umsetzung}: \textbf{MMU} (\emph{memory management unit}) setzt virtuelle Adressen in physikalische um
\end{items}

\textbf{Virtueller Speicher}
\begin{items}
  \item Speicherkapazität größer als effektive Hauptspeicherkapazität
  \item Betriebssystem lagert nach Bedarf Speicherbereiche ein/aus
  \item MMU-Adressberechnung hardwaremäßig eindeutig
  \item Abbildungsinformation in Übersetzungstabellen gespeichert
  \item \( \leadsto \) Abbildungsinformation für zusammenhängende Adressbereiche, um Übersetzungstabellen klein zu halten
\end{items}

\textbf{Virtueller Speicher -- Verwaltung (Segmentierung)}
\begin{items}
  \item Virtueller Adressraum wird in Segmente verschiedener Länge zerteilt
  \item Mehrere Segmente pro Programm (zB für Programmcode, Daten)
  \item Segmente enthalten logisch zusammenhängende Informationen, relativ groß
  \item \underline{Vorteile}: \\*
    - spiegelt logische Programmstruktur wieder \\*
    - große Segmente \( \leadsto \) relativ seltener Datentransfer
  \item \underline{Nachteile}: \\*
    - Datentransfer umfangreich falls notwendig \\*
    - Programm aus nur einem Code- und Datensegment \\* \phantom{-} \( \leadsto \) muss vollständig eingelagert werden
\end{items}

\textbf{Virtueller Speicher -- Verwaltung (Seiten)}
\begin{items}
  \item logischer und physikalischer Adressraum in Teile fester länge (Pages) zerteilt
  \item Pages relativ klein (256-4k Byte)
  \item Viele Seiten pro Prozess, keine logischen Zusammenhänge
  \item \underline{Vorteile}: \\*
    - kleine Seiten \( \leadsto \) nur wirklich benötigter \\* \phantom{-} Programmteil wird eingelagert \\*
    - geringerer Verwaltungsaufwand als Segmentierung
  \item \underline{Nachteile}: \\*
    - häufigerer Datentransfer als bei Segmentierung
\end{items}